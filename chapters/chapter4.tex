\chapter{Localization Design\label{cha:chapter4}}


As stated in \autoref{sub:kalman}, Kalman filter works by combining prediction step and measurement update to give an optimal estimate of the current state variable. In this paper, odometry information from motion module is used to perform the prediction step. From odometry, the information of how much distance the robot moved relative to the pose of last time stamp is provided, thus an estimate of the current position can be established. Likewise, an estimate from the measurement update is performed based on what the robot has observed, or simply put, the vision results from the cameras. In the end, Kalman filter update the state variable by a weighted average of the estimates.

The choice of motion model and observation model is fundamental for both Kalman filter's prediction step and measurement update. The more precise and comprehensive the model describes the system, the more accurate the estimate of the state will be. 

\section{Motion Update}\label{sec:Motion Update}
In general, motion models can be categorized by two kinds: velocity motion model and odometry motion model. Practical experience suggests that odometry, while still erroneous, is usually more accurate than velocity. Both suffer from drift and slippage, but velocity additionally suffers from the mismatch between the actual motion controllers and its mathematical model \cite{thrun2005probabilistic}. It is especially true for humanoid robot like NAO, whose moving velocity is difficult to model. 

Since the transformation between the coordinate used internally by the odometry measurement and the physical world coordinate is unknown, the internal odometry measurement in this motion model is relative. 
To be specific, in the time interval $(t-1, t]$, the robot moves from a position $x_{t-1}$ to position $x_t$, and meanwhile the odometry reports us a related movement from $\bar{x}_{t-1} = (\bar{x}, \bar{y}, \bar{\theta})$ to $\bar{x}_{t} = (\bar{x}', \bar{y}', \bar{\theta}')$. The bar here indicates that these are odometry measurements. We use the relative difference of $\bar{x}_{t-1}$ and $\bar{x}_{t}$ as an estimation of the difference between the true position $x_{t-1} = (x, y, \theta)$ and $x_t = (x', y', \theta')$. Therefore, the odometry information $u_t$ can be given by the pair:

\begin{equation}\label{eq:ut}
u_t = \begin{pmatrix}
\bar{x}_{t-1}\\ 
\bar{x}_{t}
\end{pmatrix}
\end{equation}

In our motion model, to obtain the relative odometry change, $u_t$ can be treated as a translation and then followed by a rotation. \autoref{fig:motion_model} demonstrates the decomposition of the odometry measurement. $\delta_{trans}$ is the translation and $\delta_{rot}$ is the rotation after translation. Both the translation and the rotation are considered under the coordinate $O$, which is relative to $\bar{x}_{t-1}$. The relative translation and rotation is calculated using \autoref{alg:substract poses}.

\begin{figure}[h!]
  \centering
  \includegraphics[width=.6\textwidth]{motion_modela.png}
  \caption{Relative odometry change. Translation $\delta_{trans}$, rotation $\delta_{rot}$ are considered in coordinate system $O$}
  \label{fig:motion_model}
\end{figure}

\begin{algorithm}                      
  \caption{substract\_poses ($\bar{x}_{t-1}, \bar{x}_{t}$)}         % give the algorithm a caption
\label{alg:substract poses}                           
\begin{algorithmic}[1]  
\State $\begin{bmatrix}
\delta_{x}\\ 
\delta_{y}
\end{bmatrix}
=
\begin{bmatrix}
\cos{\bar{\theta}} & \sin{\bar{\theta}}\\ 
-\sin{\bar{\theta}} & \cos{\bar{\theta}}
\end{bmatrix}\cdot 
\begin{bmatrix}
\bar{x}' - \bar{x}\\ 
\bar{y}' - \bar{y}
\end{bmatrix}$

\State $\delta_{rot} = \bar{\theta}' - \bar{\theta}$
\State $return (\delta_{x}, \delta_{y}, \delta_{rot})$

\end{algorithmic}
\end{algorithm}

Once the estimated translation $\delta_{trans}$ and rotation $\delta_{rot}$ relative to pose $x_{t-1}$ is obtained, it can be applied to update the robot position in the physical world coordinate. The full motion update using odometry measurement is shown in \autoref{alg:motion_update}.

\newcommand{\transfunc}{odometry\_motion\_update}

\begin{algorithm}                      
  \caption{\transfunc ($x_{t-1}$, $u_t$)}         % give the algorithm a caption
\label{alg:motion_update}                           
\begin{algorithmic}[1]                    
  %\State $\delta_{x}, \delta_{y}, \theta_{r} \gets \Call{sub_pose}{\bar{x}_{t-1}, \bar{x}_{t}}$
  \State $(\delta_{x}, \delta_{y}, \delta_{rot}) = substract\_poses (\bar{x}_{t-1}, \bar{x}_{t})$
  \State $
\begin{bmatrix}
  x'\\ 
  y'\\ 
\end{bmatrix}
= 
\begin{bmatrix}
  \cos\theta  & -\sin\theta \\ 
  \sin\theta &  \cos\theta  
\end{bmatrix}\cdot 
\begin{bmatrix}
  \delta_{x}\\ 
  \delta_{y} 
\end{bmatrix} +
\begin{bmatrix}
  x\\ 
  y 
\end{bmatrix}\
$
\State $\theta' = \theta + \theta_{rot}$ 
  %\State $\textit{stringlen} \gets \text{length of }\textit{string}$
\State $return [x', y', \theta']^\top$
\end{algorithmic}
\end{algorithm}

The function \textit{\transfunc} in \autoref{alg:motion_update} corresponds to the transition function $f(x_{k-1}, u_{k})$ described in \gls{EKF} algorithm in \autoref{tab:ekf}. Then the related Jacobian Matrix $F_t$ can be calculated by \autoref{eq:jacobianF}.
\begin{equation}\label{eq:jacobianF} 
F_t = 
\begin{bmatrix}
1 & 0 & -sin(\theta)\cdot \delta_x - cos(\theta)\cdot\delta_y\\ 
0 &  1& cos(\theta)\cdot \delta_x  - sin(\theta)\cdot\delta_y\\ 
0 & 0 & 1
\end{bmatrix} 
\end{equation}

The process noise covariance matrix $Q_t$ is modeled to be positive proportional to the absolute change of odometry $(|\delta_{x}|, |\delta_{y}|, |\theta_{r}|)$. Since the more the robot moves in a time interval, the more it suffers from unpredictable noises like slippage and drift. \autoref{eq:Qt} is proposed to calculate process noise covariance matrix.
\begin{equation}\label{eq:Qt} 
Q_t = 
\begin{bmatrix}
|\delta_x| & 0 & 0\\ 
0 &  |\delta_y|& 0\\ 
0 & 0 & |\delta_{rot}|
\end{bmatrix} 
\cdot
Sc^2 
\cdot
\begin{bmatrix}
|\delta_x| & 0 & 0\\ 
0 &  |\delta_y|& 0\\ 
0 & 0 & |\delta_{rot}|
\end{bmatrix}^\top
\end{equation}

\begin{equation}\label{eq:Sc} 
Sc = \begin{bmatrix}
0.8 & 0.2 & 0.2\\ 
0.2 &  0.8 & 0.2\\ 
0.2 & 0.2 & 0.8
\end{bmatrix}
\end{equation}

$Sc$ is a scaling matrix and $Sc^2$ is square elementwise. It intuitively means, when robot walk \SI{1}{\meter} in x or y direction, it may have an error range in $\pm$ \SI{0.8}{\meter}. When it rotates \SI{1}{\radian}, it may have an error of $\pm$ \SI{0.8}{\radian} in orientation. The value of $Sc$ depends on the accuracy of odometry measurement from the motion module, and a better tuning of the values can be argued.

With all the essential elements for motion model ready, we plug in the formula from \autoref{tab:ekf} and perform the \gls{EKF} prediction step for the localization algorithm. The only thing we haven't touched yet is the initial value of $x_{t-1}$ and $P_{t-1}$, it will be discussed in ?(to be determined).  



\section{Sensor Update}\label{sec:sensor update}
The robot can also measure its postion by sensing the surrounding environment. In the case of NAO robot, the measurement is made by observing the environment from its two cameras. The images from the camera is preprocessed by the DAInamite's vision module to extract the landmarks as shown in \autoref{fig:perception}. During the work of this thesis, two sensor update models are proposed to accomplish the localization task. One is based on an optimization approach and another is based on features. We will discuss both model in this section.

\subsection{Optimzation Based Model}\label{sec:optimization update} 
In the early phase of the thesis, the only unqiue landmark can be detected by vision module is the yellow goal posts, however, from 2015 the yellow goal post will be replaced by white goal post according to the rule \cite{Committee2013}. Then the only landmark available is the white line in the field. Field lines are of high ambiguity, in the case of feature based Kalman filter localization, a wrong matching of the landmarks could lead to the divergence of the tracking which can hardly be recovered. Thus, a robust and accurate algorithm for sensor update is implemented taking the reference from Lauer \cite{Lauer2006}. The algorithm is based on Rprop \cite{Riedmiller1993} optimization method,  which uses only the line points from the detected lines, to find the robot postion which match the field map best. 


\subsubsection{Image Frame to Robot Frame}

\begin{figure}[h!]
  \centering
  \includegraphics[width=.8\textwidth]{camera-matrix.png}
  \caption{coordinate of the bottom camera and the robot local coordinate}
  \label{fig:motion_model}
\end{figure}


\subsection{Feature Based Model}\label{sec:feature update} 
Altough the optimization method based localizaiton outperforms the particle filter in terms of accuracy, it is not computational efficient. During the preparation of German Open 2015, DAInamite team developed more feature detections like center circle detection, penalty area detection and L corner detection. The richness of features motivates me to develop feature based sensor update for localization.  
