\thispagestyle{empty}
\chapter*{Abstract}
%\vspace*{1.0cm}

%\begin{center}
%    \textbf{Abstract}
%\end{center}
%
%\vspace*{0.5cm}

%\noindent

Self-localization is a crucial part for autonomous robot, particularly in the RoboCup \gls{SPL}. It is the prerequisite for the robot to accomplish the remaining decision making tasks. In short, the robot needs to know where it is in the game field. The challenge lies in the domain of localization is that robots have only partial observations about the surrounding world, they need to fuse sensor information to estimate its location. Many localization algorithms have been proposed in the past years. Most of them are variants based upon Bayesian theories, the two widely used algorithms are particle filters and multi-model Kalman filters. Although particle filters are robust, multi-model Kalman filters outperforms it in terms of accuracy. In this thesis, an efficient localization algorithm will be investigated by combining the essence of particle filter and the Kalman filter to gain better performance for localization and implemented for NAO robot in the RoboCup \gls{SPL}.
